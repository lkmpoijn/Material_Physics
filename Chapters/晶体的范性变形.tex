\chapter{晶体的范性变形}
    \section{单晶体的滑移变形}
        \subsection{滑移晶体学特征}
        \subsection{影响滑移系统的因素}
        \subsection{滑移方式与滑移带}
    \section{单晶体屈服与晶体的转动及碎化}
        \subsection{临界分切应力定律}
        \subsection{临界切应力的位错理论}
            晶体收到外应力作用,取向因子最大的滑移系位错开始滑移,其他位错或晶体缺陷要对它的运动产生
            阻碍或交互作用,主要考虑以下几种
            \begin{enumerate}
                \item[1] 位错增殖;
                \item[2] 点阵阻力;
                \item[3] 与其他位错的交互作用:
                \begin{enumerate}
                    \item[1] 弹性应力场的交互作用。为了简便起见,设有二个位错排列在垂直的方向,相距为l。位错要从B、C中间穿过去,就要克服B、C的长程弹性作用力。假设平行于可动位错的那些位错均匀分布,这些位错之间的间距用$l_1$表示,可以推导得出:位错克服的应力为$\tau=\alpha\mu b\sqrt{\rho}$。
                    \item[2] 位错塞积;塞积导致应力集中\footnote{如果塞积群与所求位错之间距离较远,可以视塞积群为一个扩大$N$倍的柏式矢量,其中$N$为塞积群位错数量。}。位错受到的阻力为$\tau_0^2=\alpha\frac{\mu b}{l_2}$
                    \item[3] 位错绕过位错林;与增殖过程相似,在林位错周围形成位错环,然后继续滑移,阻碍切应力为$\tau_0^3=\alpha\frac{\mu b}{l_2}$;
                    \item[4] 以上的三个$\alpha$并非同一常数,仅仅为方便使用同一符号,而且以上的阻力与位错的温度没有关系,
                    \item[5] 切过林位错产生交割,这一部分是短程相互作用,与热激活有关,这是温度升高,屈服强度降低的原因。外力为$\tau$,长程应力为$\tau_0$,外力完成完全切割做功为$W=(\tau-\tau_0)bld$,产生割阶的能量为$\Delta H_0$,热激活提供的能量为$\Delta H=\Delta H_0-(\tau-\tau_0)bld$,当温度高于一定值,形成割阶的能量完全由热激活提供,不需要外力。
                \end{enumerate} 
            \end{enumerate}
            提高屈服强度也就是提高位错运动的长程和短程阻力,主要有以下途径:
            \begin{itemize}
                \item[1] 位错密度上升,比如通过拉拔进行加工硬化,桥梁钢在处理后屈服强度可提升50\%;
                \item[2] 障碍物尺寸增加;
                \item[3] 障碍物间距缩小,也就是增加障碍物密度,比如弥散强化;
                \item[4] 障碍物的稳定性增加。
            \end{itemize}
            影响临界分切应力$\tau_0$的因素
            \begin{itemize}
                \item[1] 高温段不发生变化,温度降低,应力上升,对韧性产生影响;
                \item[2] 合金元素的存在,一般会使临界切应力$\tau_0$增加;
                \item[3] 应变使位错密度增加,进而也会使切应力增加;
            \end{itemize}
        \subsection{拉伸过程中集体的转动和碎化}
            滑移面与拉伸轴不平行,切变方向沿滑移方向,无法沿拉伸轴方向,因此,试样两端会产生位移。在拉伸实验中,两端被限制,晶面会受到转动力矩的作用,从而发生转动。

            在不能整体发生倾转的时候,表面出现$S$状的滑移线,同号的位错堆积起来,然后产生晶体弯曲,变形过程中晶体的碎化,在取向差小于\ang{15}时\footnote{小角晶界处。},将会出现亚晶。
            产生畸变后,在衍射斑上会出现星芒。
        \subsection{孪生}
            孪生是晶体塑性变形的另一种方式,比如,体心立方的金属的六个偏位错滑移产生孪晶,当晶体的一部分相对另一部分呈镜面对称时,两者互为孪晶。
            
            其与滑移的差异
            \begin{itemize}
                \item[1] 原子到孪生面的移动距离不是常数;
                \item[2] 抛光之后仍然可见;而滑移是表面现象,内部不产生畸变,而孪生不会因为抛光改变晶体排列,仍然可见;
                \item[3] 形状通常为薄透镜状;
                \item[4] 发生速度可以很快;
                \item[5] 一般是滑移受阻时产生的。
            \end{itemize}
        \subsection{总结}
            单晶体塑性变形的三个过程:1.切变,2.转动,3.碎化。
    \section{多晶体范性变形的特点及晶界的作用}
        \subsection{多晶体塑性变形的特点}
            一般情况下,多晶材料屈服应力大于单晶材料,因为晶界会产生强化效应。
            在若干晶系中,选择$\omega$最大的取向最先发生开动,晶粒中发生多滑移,使其发生变形的方向一致。二这需要至少五个滑移系同时运动。
            这也是多晶变形的特点之一:多滑移。

            在晶界附近,为位错滑移受到阻碍,发生位错塞积。

            在变形过程中,会出现择优取向,也就是织构。
        \subsection{多晶体的屈服应力}
            假设在A 晶粒中的某个滑移系处于有利的取向,其取向因子最大,也最先开始滑
            移,滑移后位错遇到晶界,位错塞积,之后在B晶粒的晶界处产生应力集中,当应力集中过大时,
            B中的滑移系达到了临界应力,也会发生滑移,变形也就从$A$传递到了$B$。这是变形的传递过程。

            假设A晶粒中的变形可以表示为
            \begin{equation}
                \gamma=\frac{nb}{L},
            \end{equation}
            其中$L$为A晶粒的滑移方向的长度,
            此时B晶粒没有发生位错滑移,仍然处于弹性变形阶段,所以

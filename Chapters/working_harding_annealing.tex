\chapter{加工硬化与退火}
    根据霍尔佩奇公式可知,对于多晶材料,当晶粒极端细化时,屈服强度会达到无限大,然而这是不可能的。
    后来根据一些经验,人们认为该公式中应当分段。

    从拉伸曲线上看,当应力大于屈服应力时,应变增加,应力也会增加,这一现象称为加工硬化\index{加工硬化}。
    加工硬化是结构材料的必备性质,使材料的强度得到提高,其益处在于:
    \begin{itemize}
        \item[1] 防止材料被进一步破坏;
        \item[2] 同时保证了材料的均匀变形能力。
    \end{itemize}
    同样也有着一定的坏处
    \begin{itemize}
        \item[1] 加工过程需要退火,变形逐渐困难;
        \item[2] 加工硬化后,材料的韧性降低;
        \item[3] 由于加工硬化能力不足,材料可能会出现缩颈;
    \end{itemize}
    在描述这一性质时有一些参数需要先介绍:
    \begin{itemize}
        \item[1] $n$:加工硬化指数,其数值与最大力对应的型变量相等;
        \item[2] $r$:方形材料的横向变形与纵向变形之比,与内部的织构有关,即滑移面与板面的平行关系;
        \item[3] $\frac{\dif \sigma}{\dif \varepsilon}$:加工硬化率,其与应力应变曲线的交点即为颈缩开始点。
    \end{itemize}
    \section{单晶体加工硬化特征}
        \subsection{面心立方应力应变曲线}
            面心立方晶体的形变效果较好,因此方便研究,也就是不选择体心立方和密排六方的原因。
            实验采用\textbf{单晶面心立方晶体}的加工硬化曲线可以分为三个阶段
            \begin{itemize}
                \item[1] I区:$\dif\tau/\dif\varepsilon$加工硬化率较小,容易滑移,数值大约为$2\time10^{-4}\mu$;
                \item[2] II区:$\dif\tau/\dif\varepsilon$数值基本为常数,大小为$3\times10^{-2}\mu$,称为线性硬化区;
                \item[3] III区:$\dif\tau/\dif\varepsilon$逐渐变小趋于0,称为抛物线硬化区。
            \end{itemize}
            在多晶中难以观察到第一阶段。

            研究发现,有以下因素
            \begin{itemize}
                \item 温度上升使临界分切应力$\tau_c$下降,第一阶段和第二阶段分界点以及第二第三阶段分界点$\gamma_2$和$\gamma_3$显著下降,从而第二第三阶段的切应力都会显著下降;
                \item 层错的影响:拓展位错的宽$d$\footnote{拓展位错宽度增加,层错能下降,交滑移难度增加。}与拓展位错的交滑移,层错能下降,第2第3阶段影响大;
                \item 取向对影响第二阶段分界点和第一部分的斜率;
                \item 合金元素可能是第一阶段增长。
            \end{itemize}

            从滑移线分析,
            \begin{itemize}
                \item [1]第一阶段滑移线较细,在光学显微镜一般观察不到,而且都是相同的指数;
                \item [2]第二阶段的滑移线有交叉,开始双滑移,线长度随型变量的增加而逐步变短;
                \item [3] 第三阶段的滑移带不再连续,位错呈现胞状结构\footnote{与回复阶段有关}。
            \end{itemize}
        \subsection{体心和六方应力应变曲线}
            高纯度的体心立方和六方晶体也是三个阶段,第三阶段与拓展位错交滑移有关,由于体心立方的层错能大,
            因此不容易观察到第三阶段。而观察六方的三阶段更是要求取向合适,因而观察困难。
    \section{加工硬化的位错理论}
        加工硬化的理论主要有两种类型:
        \begin{itemize}
            \item[1] 平行位错硬化理论:与临界分切应力位错理论相似,认为主滑移线的平行位错阻力的长程应力增加;
            \item[2] 交截位错硬化理论:认为林位错对主滑移系产生阻碍,是短程的硬化理论;
        \end{itemize}
        \subsection{第一阶段的位错理论}
        第一阶段位错沿滑移面运动,基本不与其他位错发生交互作用。这一阶段主要是
        位错发生单滑移,
        \begin{equation}
            \tau=\alpha\mu b\rho^{-\frac{1}{2}},
        \end{equation}
        位错源开动,在L-C位错锁\index{L-C位错锁}\footnote{L-C位错锁与双滑移有关,第一阶段为单滑移,L-C位错锁数量不会增加。}附近发生塞积,位错的密度有少量增加,切应力增加。
        假设不动位错的数量为$N''$,单位面积不动位错前塞积的位错数目是有上限的,用$n$表示
        塞积的总位错数为$N''*n$,所以切应力为
        \begin{equation}
            \tau=\alpha\mu nb\sqrt{N''},
        \end{equation}
        此时的变形为
        \begin{equation}
            \gamma=\rho Sb=N''L(nb),
        \end{equation}
        其中$L$为位错长度。
        
        此时的加工硬化率为
        \begin{equation}
            \theta=\frac{\dif\sigma}{\dif \varepsilon}=\alpha\mu\frac{1}{L}\cdot N''
        \end{equation}

        \subsection{加工硬化的第二阶段的理论}
            位错开始发生双滑移和多滑移,位错塞积在L-C位错锁前形成新的位错环。由于仍然是塞积在位错锁,
            所以塞积的上限一定$n$不变,而位错锁前形成新的位错锁,所以$N$增加,也可以说是被激活。
            此时产生的形变的增加为
            \begin{equation}
                \dif \gamma=L(nb)\dif N,
            \end{equation}
            位错的长度为
            \begin{equation}
                L=\frac{\Lambda}{\gamma-\gamma_2},
            \end{equation}
            $\gamma_2$为第二阶段切应变分界点,将$L$带回后积分
            \begin{equation}
                \int(\gamma-\gamma_2)\dif\gamma=\int\Lambda(nb)\dif N,
            \end{equation}
            所以形变变化量为
            \begin{equation}
                \gamma-\gamma_2=\sqrt{2knb\Lambda N},
            \end{equation}
            此时的切应力为
            \begin{equation}
                \tau=\alpha\mu(nb)\sqrt{N},
            \end{equation}
            此时的加工硬化率为:
            \begin{equation}
                \theta_2=\frac{\tau}{\gamma-\gamma_2}=\alpha\mu\sqrt{\frac{nb}{2\Lambda}},
            \end{equation}
            
            这只是理论中的一种,第二阶段的理论仍然有较大争议。
        \subsection{第三阶段的位错理论}
            塞积在不动位错前的螺位错发生交滑移,由于正号和负号的位错相互抵消,使得加工硬化减弱,这一过程与温度的关系很大。
        \subsection{包辛格效应}
            金属在正向加载后然后反向加载,会导致屈服应力降低,在单晶体中不容易观察到,由于这一现象与
            晶界的位错塞积和残余应力有关,因此在多晶体中较为明显。

            比如金属处在高周疲劳情况,屈服应力会不断降低,最后断裂。
    
    \section{多晶体的加工硬化}
        对于多晶体来说,其加工硬化特征也可以分为三个阶段,但是与单晶体的三阶段区别在于一开始就会发生多滑移。

        多晶比单晶多了晶界,而且从滑移一开始就是多滑移。屈服过程是逐步的,没有明确的屈服点。(第一阶段和第二阶段的转折点标志着所有晶粒同时变形)而且晶界是位错运动的重要障碍,所以屈服强度高于单晶的。
        
        二、三阶段位错机制与单晶体相似,所以加大应力后多晶与单晶的加工硬化率逐步接近。
        
    \section{形变金属的加热}
        残余应力有三类,原子层面、单晶层面以及多晶层面。残余应力的大部分能量都来自位错。
        对其加热可以释放应变能,其中会发生三种现象:残余应力消除,发生多边形化,晶粒重新改组,。

        在加热过程中,残余应力消除,发生多边形化是回复的现象,晶粒重新改组为再结晶对应的现象。

        通常情况下,所有境界迁移速度相图,晶粒尺寸均匀。异常情况下,只有少数晶界可以迁移,导致个别晶体形态巨大,
        比如对Al单晶长时间退火,可以形成单晶。

        回复过程中,显微组织组织的变化几乎不可见,再结晶阶段,变形晶粒通过形核长大,逐渐转变为新的无畸变的等轴晶粒,然后等轴晶粒开始长大吸收其它晶粒。

        \subsection{性能变化}
            回复阶段,强度和硬度略有下降,塑性略有提高;再结晶阶段,强度和硬度明显下降,塑性明显提高。
            晶粒长大时,强度硬度继续下降,塑性进一步提高,粗化严重时下降,

            密度会逐渐提高,在再结晶阶段快速提高,电阻在回复阶段可明显下降。


            形变处储存能主要来自弹性应变能、点缺陷和\textbf{位错},在回复阶段开始释放,在再结晶阶段大量释放。
            形变储存能也是回复和再结晶的驱动力。

            回复阶段第一种内应力可以消除,再结晶阶段全部消除。

        \subsection{回复动力学}
            部分材料可以在室温阶段发生回复过程,但是大部分材料都要在较高温度下发生。

            回复过程是动态过程,可以在形变过程中发生,速度较快,时间较短可以发生部分回复,
            加载间隔时间较长可以发生完全回复。

            回复动力学曲线特点
            \begin{itemize}
                \item[1] 没有孕育期;
                \item[2] 开始变化快,逐渐变慢;
                \item[3] 长时间处理后,趋于一个平衡值。
            \end{itemize}

            回复分为根据温度分为低温回复、中温回复和高温回复,以下用$T_m$代表熔点。

            低温恢复发生在0.3$T_m$以下,主要由点缺陷产生
            \begin{itemize}
                \item[1] 空位移到缺陷、位错的地方;
                \item[2]  空位和自间隙原子消失;
                \item[3] 空位发生聚集,
            \end{itemize}
            最后导致缺陷的密度降低,减少弹性畸变能。

            中温回复在$0.3-0.5T_m$发生,异号位错相遇而抵消或者是位错缠结发生重排,因此中温回复的特点为位错密度明显降低。

            高温回复发生在$0.5T_m$以上,位错攀移导致位错垂直排列,产生亚晶界,
            多晶化产生亚晶粒,弹性畸变能降低。
        
            回复的影响因素主要有
            \begin{itemize}
                \item[1] 温度;
                \item[2] 层错能:层错能较低时,攀移位错受阻,难以形成亚晶,或形成不完整亚晶;
                \item[3] 变形量:受到变形越大,
                \item[4] 纯度:纯度越高,再结晶越难发生,回复越易发生。
            \end{itemize}

            主要应用于去应力退火,降低应力,保持加工硬化效果,防止工件变形、开裂,提高耐蚀性。

        \subsection{再结晶}
            形变金属\textbf{加热到一定温度}后,新的无畸变晶粒\textbf{消耗掉冷加工的畸变晶粒}的\textbf{形核与长大过程}。

            新旧晶粒结构相同,因此再结晶不是相变,但是取向完全不同,性能恢复到形变之前。

            发生再结晶的条件是变形量大于临界变形量,这样才能储存足够的应变能。而影响再结晶
            最终的晶粒尺寸与变形量、退火温度以及原始晶粒度有关。

            再结晶温度:经严重冷变形(变形量>70\%)的金属或合
            金,在1h内能够完成再结晶的(再结晶体积分数>95\%)
            最低温度。
            
            在工业中,都是确定变形量和完成再结晶时间固定后才能确定再结晶温度,也就是条件再结晶温度,或称为再结晶停止温度\index{再结晶温度!再结晶停止温度}。

            根据经验公式,纯度越高,越难以发生再结晶,应用时再结晶退火温度比再结晶温度高\SI{100}{\celsius}。

            再结晶的形核过程与相变中的形核过程不同,有两个机制:
            \begin{itemize}
                \item[1] 变形量较大时,在回复过程中由亚晶合并形核,靠亚晶移动形核;
                \item[2] 变形量较小时,大角晶界凸出形核,完全靠晶界迁移,晶核伸向畸变能较高的区域。
            \end{itemize}

            再结晶过程存在孕育期,温度越高,变形量越大孕育期越短;在体
            积分数为0.5时速率最大,然后减慢。

            再结晶温度的影响因素有:
            \begin{itemize}
                \item[1] 变形量越大,驱动力越大,再结晶温度越低;
                \item[2] 纯度越高,再结晶温度越低;
                \item[3] 加热速度太低或太高,条件再结晶温度提高。
            \end{itemize}
            
            对再结晶的影响因素有
            \begin{itemize}
                \item[1] 退火温度。温度越高,再结晶速度越大;
                \item[2] 变形量越大,再结晶温度越低;随变形量增大,再结晶温度趋于稳定;变形量低于一定值,再结晶不能进行;
                \item[3] 原始晶粒尺寸。晶粒越小,驱动力越大;晶界越多,有利于形核。
                \item[4] 微量溶质元素阻碍位错和晶界的运动,不利于再结晶;
                \item[5] 分散第二相,间距和直径都较大时,提高畸变能,并可作为形核核心,促进再结晶;直径和间距很小时,提高畸变能,但阻碍晶界迁移,阻碍再结晶。
            \end{itemize}

            再结晶应用于恢复变形能力、改善显微组织、消除各向异性以及提高组织稳定性。
        \subsection{晶粒长大}
            \subsubsection{正常长大}
            晶粒长大的驱动力为界面能变化,长大方式有两种,分别为
            \begin{itemize}
                \item[1] 正常长大:再结晶后的晶粒均匀连续的长大,驱动力为界面能差,曲率半径越小,驱动力越大,长大方向指向曲率中心;
                \item[2] 异常长大,也称为二次再结晶。
            \end{itemize}

            晶粒正常长大方向是指向曲率的中心,曲率半径越小,驱动力越大,再结晶晶核的长大方向相反。

            晶粒稳定条件是晶界夹角为\ang{120},边界趋于平直,二维坐标的晶粒边数趋于6。

            影响因素有
            \begin{itemize}
                \item[1] 温度:温度越高,晶界易迁移,晶粒易粗化;
                \item[2] 分散相粒子:阻碍晶界迁移,降低晶粒长大速率;
                \item[3] 杂志和合金元素:气团钉扎晶界,不利于晶界移动;
                \item[4] 晶粒的取向差:小角度晶界能小于大角度晶界,前者的移动速率低于后者。
            \end{itemize}

            \subsubsection{晶粒异常长大}
                异常长大是指:少数再结晶晶粒的急剧长大现象。正常的晶粒长大过程被第二相或织构限制。
                驱动力主要是界面能的变化。

                有三种机制:1. 钉扎晶界的第二相溶于基体;2.再结晶织构中位向一致晶粒的合并;
                3. 大晶粒吞并小晶粒。

                对组织和性能的影响是
                \begin{itemize}
                    \item[1] 织构明显,出现各向异性,磁导率强化;
                    \item[2] 晶粒大小不均匀,整个材料的性能不均匀;
                    \item[3] 晶粒粗大,降低了材料的强度和塑性,提高了表面的粗糙度。 
                \end{itemize}

                在金属的轧制过程中,远高于再结晶温度的轧制以及远低于再结晶温度不会产生混晶\footnote{个别晶粒的大小远大于周围晶粒的尺寸。},
                否则会对材料的韧性产生较大危害,尤其在断裂问题中,一个较大晶粒非常容易传导裂纹。

            \subsubsection{再结晶退火的组织}
                首先要参考再结晶图了解退火温度、变形量与晶粒大小的关系图,可能会产生再结晶织构,也就是退火过程中
                形成的新织构;也有可能形成退火孪晶,在面心立方中常见,由于晶界迁移出现层错引起的。

            \subsubsection{动态回复与动态再结晶}
                动态回复是指在塑性形变中发生的回复,拉伸曲线会波动向上;动态再结晶是指在变形过程中发生
                动态再结晶,拉伸曲线在某一值附近波动。

                组织有两个特点:
                \begin{itemize}
                    \item 反复形核,有限长大,最后的晶粒非常细小;
                    \item 包含亚晶粒,位错密度较高,强度和硬度都很高。 
                \end{itemize}

                在生产中采用低的变形终止温度、较大的最终变形量(平均一道次变形量一般要大于70\%),快的冷却速度可获得细小晶粒。
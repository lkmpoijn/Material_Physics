\chapter{金属的断裂}
    材料在实际的使用过程中,存在一系列的构件失效的现象,其中最为重要的一种就是断裂。
    断裂过程往往突然发生,这会对使用造成巨大的危害,而且无法预期。在这里关注断裂的物理本质。
    对化学问题不做研究。

    断裂在金属材料中被认为是金属材料在变形超过其塑性极限,原子间结合力被破坏,
    结构完全分开。
    断裂的核心问题是研究断裂从何处出现和裂纹如何发展。
    \section{断裂的基本类型}
        根据断裂前金属是否有明显的塑性变形可以分为脆性断裂和韧性断裂,
        一般认为如果变形量$\varepsilon<5\%$就是脆性断裂。

        对于断口的分类有很多种:
        \begin{itemize}
            \item 按照断裂面相对作用力的取向关系可以分为正断和剪断;
            \item 从微观上裂纹的走向可以分为穿晶断裂和沿晶断裂(冰糖断口);
            \item 从晶体学的特征上可以分为切变断裂和解理断裂;
            \item 断口形貌上还分为纤维状断口、结晶状断裂、疲劳断裂和应力腐蚀断口。
        \end{itemize}

        \subsection{解理断裂}
            解理断裂是指当应力达到某一临界值时沿一特定结晶学方向发生的突然断裂\footnote{考试题:三条重点。},
            宏观特征为断口平滑光亮、一般在体心、六方中易于出现。 温度越低、应变速度越大易于出现。
            电子显微镜下的形貌为河流状花样。解理面一般是原子间距最大的晶面,因为此处原子结合键最弱。

        \subsection{韧性断裂}
            经过明显的变形后发生的断裂称为韧性断裂。拉伸时以颈缩为先导,当应变硬化产生的强度增加不足以补偿截面积的减少时,
            产生应力集中变形,出现“细颈”。

            细颈中心承受三向拉应力, 显微空洞首先在此形成, 随后长大聚合成 裂纹, 最终在细颈边缘处,沿与拉伸轴\ang{45}
            方向被剪断,形成“杯锥”断口,在微观形态下,由于夹杂物的存在产生空洞,在拉伸
            过程中产生韧窝。

            韧性断裂的特点:
            \begin{itemize}
                \item 断裂前发生较大塑性变形,吸收大量的能量;
                \item 裂纹产生然后扩展最后聚合;
                \item 裂纹扩展临界应力大于裂纹形核应力,导致裂纹缓慢增加。
            \end{itemize}
    \section{脆性断裂}
        \subsection{理论断裂强度}
            完整晶体在正应力作用下沿某一晶面拉断的强度。两相邻原子面在拉力作用下,克服原子间键合力
            作用,是原子面分开的应力。

            根据原子间作用模型,原子间作用力与位移间的关系满足
            \begin{equation}
                \sigma=\sigma_{m}\sin\left( \frac{2\pi x}{\lambda} \right)
            \end{equation}
            将原子拉开所需的最大应力,即断裂理论强度。
            将两个原子面拉开所做的功为
            \begin{equation}
                \int_{0}^{\frac{\lambda}{2}}\sigma_m\sin\left( \frac{2\pi x}{\lambda} \right)\dif x=\frac{\lambda\sigma_m}{\pi},
            \end{equation}
            断裂后产生两个断裂面,表面能为$2\gamma$,
            外力抵抗原子间结合力所作的功为产生断裂新面的表面能
            \begin{equation}
                \frac{\lambda\sigma_m}{\pi}=2\gamma,
            \end{equation}
            与虎克定律联立解得
            \begin{equation}
                \sigma_m=\left( \frac{\gamma E}{a} \right)^{1/2},
            \end{equation}
            对于一般金属,计算得到的理论断裂强度为\SI{10000}{\MPa},这与实际仅为\SIrange{200}{500}{\MPa}
            的断裂强度不符,因此完整晶体模型在这里仍然是不适用的。
        \subsection{Griffith脆断理论}
            Griffith认为材料在断裂之前已经存在微裂纹,从裂纹尖端引起应力集中,在外加应力小于理论断裂强度
            时裂纹扩展,实际裂纹强度将大大降低。

            理论认为,弹性能减少不仅仅用于材料变形,裂纹长度的增加也消耗了一部分弹性能。根据
            这一理论计算得到的最大断裂强度为
            \begin{equation}
                \sigma_c=\left( \frac{2E\gamma}{\pi c} \right)^{\frac{1}{2}}\label{Griffth脆断},
            \end{equation}
            其中$c$为晶体中存在的裂纹长度的一半。相当于裂纹两端引起的应力集中把外力
            放大了$\left( \frac{c}{a} \right)^{1/2}$倍,使局部达到了理论断裂强度。
        
            这个理论虽然解释了理论断裂强度与实际的差别,但是其中还是省略一部分,
            比如忽略了脆性断裂之前的塑性形变,因此Orowan对该理论提出了修正公式,
            考虑了塑性变形能
            \begin{equation}
                \sigma_c=\left[ \frac{2E(\gamma+P)}{\pi c} \right]^{\frac{1}{2}},P\gg\gamma,
            \end{equation}
            其中$p$为断口表面的塑性应变能。
        \subsection{裂纹形核}
            运动的位错遇到了某种障碍, 就产生了应力集中,应力大到可以破坏原子间的键合力时,裂纹开始形核,裂纹长大导致断裂。
            
            形核机制有几种:
            \begin{itemize}
                \item[1] 位错塞积,在相界处发生应力集中;
                \item[2] 位错反应,在两相交的滑移面上,由于位错反应发生了同号位错的聚合便产生了微裂纹;
                            比如在体心立方的$(101)$面上发生如下位错反应
                            \begin{equation}
                                \frac{a}{2}[\bar{1}\bar{1}1]+\frac{a}{2}[111]=a[001],
                            \end{equation} 
                            生成的新位错为不滑动刃型位错,其柏氏矢量与$(001)$解理面垂直,形成解理裂纹;
                \item[3] 位错墙侧移理论:热型位错垂直排列,形成位错墙,在滑移面发生弯折,在外力下,位错墙侧移,在滑移面上生产裂纹;
                \item[4] 位错销毁理论:异号刃型位错相对运动,彼此合并形成空隙,以此产生裂纹。
            \end{itemize}
        \subsection{裂纹传播}
            以塞积位错造成裂纹形核为例,产生新的表面和位错的塞积能需要消耗能量,而外力做功和弹性能释放为此提供能量。

            假设形成裂纹时的应力就是屈服应力 滑移带的长度等于晶粒直径$d$,断裂强度
            \begin{equation}
                \sigma_c=\frac{2\mu\gamma}{k} d^{-\frac{1}{2}},
            \end{equation}
            说明晶粒越细,传播阻力越大。由于晶粒取向不同,穿过晶界导致传播转向。这也是提高韧性的方法。
            而且晶粒尺寸越小,试样内存在着道道临界尺寸的裂纹的几率越小,当晶粒尺寸为\SI{1}{\micro\m}时
            断裂强度就可以接近理论断裂强度。
    \section{金属断裂方式的转变}
        \subsection{影响断裂类型的因素}
            在一定条件下,材料可以从韧性断裂转变为脆性断裂。而影响转变的因素有
            有变形温度、变形速率、应力状态和组织结构。

            这里主要关注温度的影响,定义材料的韧脆转变温度为从韧性断裂到脆性断裂的转变温度称为脆性
            转变温度$T_c$。
            
            温度对于断裂应力$\sigma_f$的影响不大,然而温度升高会使材料的屈服强度降低,因此
            温度高于$T_c$,材料的断裂应力$\sigma_f$大于屈服应力$\sigma_s$,因此发生塑性断裂;温度低于$T_c$时
            则发生脆性断裂。
            
        \subsection{影响材料韧脆转变温度的因素}
            材料的韧脆转变温度与
            \begin{itemize}
                \item[1] 结合键;
                \item[2] 晶体结构,这与滑移系和变形能力有关,比如奥氏体合金钢可以在低温的条件下使用;
                \item[3] 有序度,目前关于有序度的结论还是很少。
            \end{itemize}
            这三个方面是材料的基本特性,难以改变,但是改变材质却是非常容易调控的方法。
            从材质上提高材料韧性的方法有\footnote{50\%的可能出现在考试中。}
            \begin{itemize}
                \item[1] 净化:夹杂物是裂纹形核的优先位置,在钢中比如氧化物的夹杂,在工业中一般使用真空处理,降低氧和氢的含量;
                \item[2] 细化:晶粒越细,裂纹扩展越困难;
                \item[3] 变质:只能在钢体材料中使用,一般对于\ce{MnS},由于其容易变形,容易形成长条形,通过加\ce{Ca},将其变为球形,可以降低应力集中;
                \item[4] 降碳:主要原因是碳含量升高以后,马氏体和渗碳体容易形成,而且这种相比较脆,因此降低碳含量可以提高韧性;
                \item[5] 其它方法。
            \end{itemize}
    \section{断裂韧性的概念及其应用}
        \subsection{早期工程设计}
            早期的设计主要使用强度设计,使结构材料的屈服强度高于计算的载荷,但是在实际工程中难以满足,
            由于实际的载荷可能更为复杂。在了解了载荷的情况后,后来逐渐出现了韧性要求。

            但是通过小试样检测的韧性难以代表整个材料的韧性,在实际的生产中也反映了这种情况,因此出现了更多的要求。
            提出了断裂韧性的概念。
        \subsection{断裂韧性的意义}
            用$K_c$代表了材料抵抗裂纹的扩展能力,根据Griffith脆断理论\autoref{Griffth脆断}可得
            \begin{equation}
                \frac{\sigma_c^2 \pi c}{2}=E\gamma=K_c,
            \end{equation}
            在无损检测中经常使用,根据一般分为三种,更为具体的知识不在这里继续讲述。
            
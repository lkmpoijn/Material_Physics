\chapter{加工硬化与退火}
    根据霍尔佩奇公式可知,对于多晶材料,当晶粒极端细化时,屈服强度会达到无限大,然而这是不可能的。
    后来根据一些经验,人们认为该公式中应当分段。

    从拉伸曲线上看,当应力大于屈服应力时,应变增加,应力也会增加,这一现象称为加工硬化\index{加工硬化}。
    加工硬化是结构材料的必备性质,使材料的强度得到提高,其益处在于:
    \begin{itemize}
        \item[1] 防止材料被进一步破坏;
        \item[2] 同时保证了材料的均匀变形能力。
    \end{itemize}
    同样也有着一定的坏处
    \begin{itemize}
        \item[1] 加工过程需要退火,变形逐渐困难;
        \item[2] 加工硬化后,材料的韧性降低;
        \item[3] 由于加工硬化能力不足,材料可能会出现缩颈;
    \end{itemize}
    在描述这一性质时有一些参数需要先介绍:
    \begin{itemize}
        \item[1] $n$:加工硬化指数,其数值与最大力对应的型变量相等;
        \item[2] $r$:方形材料的横向变形与纵向变形之比,与内部的织构有关,即滑移面与板面的平行关系;
        \item[3] $\frac{\dif \sigma}{\dif \varepsilon}$:加工硬化率,其与应力应变曲线的交点即为颈缩开始点。
    \end{itemize}
    \section{单晶体加工硬化特征}
        \subsection{面心立方应力应变曲线}
            面心立方晶体的形变效果较好,因此方便研究,也就是不选择体心立方和密排六方的原因。
            实验采用\textbf{单晶面心立方晶体}的加工硬化曲线可以分为三个阶段
            \begin{itemize}
                \item[1] I区:$\dif\tau/\dif\varepsilon$加工硬化率较小,容易滑移,数值大约为$2\time10^{-4}\mu$;
                \item[2] II区:$\dif\tau/\dif\varepsilon$数值基本为常数,大小为$3\times10^{-2}\mu$,称为线性硬化区;
                \item[3] III区:$\dif\tau/\dif\varepsilon$逐渐变小趋于0,称为抛物线硬化区。
            \end{itemize}
            在多晶中难以观察到第一阶段。

            研究发现,有以下因素
            \begin{itemize}
                \item 温度上升使临界分切应力$\tau_c$下降,第一阶段和第二阶段分界点以及第二第三阶段分界点$\gamma_2$和$\gamma_3$显著下降,从而第二第三阶段的切应力都会显著下降;
                \item 层错的影响:拓展位错的宽$d$\footnote{拓展位错宽度增加,层错能下降,交滑移难度增加。}与拓展位错的交滑移,层错能下降,第2第3阶段影响大;
                \item 取向对影响第二阶段分界点和第一部分的斜率;
                \item 合金元素可能是第一阶段增长。
            \end{itemize}

            从滑移线分析,
            \begin{itemize}
                \item [1]第一阶段滑移线较细,在光学显微镜一般观察不到,而且都是相同的指数;
                \item [2]第二阶段的滑移线有交叉,开始双滑移,线长度随型变量的增加而逐步变短;
                \item [3] 第三阶段的滑移带不再连续,位错呈现胞状结构\footnote{与回复阶段有关}。
            \end{itemize}
        \subsection{体心和六方应力应变曲线}
            高纯度的体心立方和六方晶体也是三个阶段,第三阶段与拓展位错交滑移有关,由于体心立方的层错能大,
            因此不容易观察到第三阶段。而观察六方的三阶段更是要求取向合适,因而观察困难。
    \section{加工硬化的位错理论}
        加工硬化的理论主要有两种类型:
        \begin{itemize}
            \item[1] 平行位错硬化理论:与临界分切应力位错理论相似,认为主滑移线的平行位错阻力的长程应力增加;
            \item[2] 交截位错硬化理论:认为林位错对主滑移系产生阻碍,是短程的硬化理论;
        \end{itemize}
        \subsection{第一阶段的位错理论}
        第一阶段位错沿滑移面运动,基本不与其他位错发生交互作用。这一阶段主要是
        位错发生单滑移,
        \begin{equation}
            \tau=\alpha\mu b\rho^{-\frac{1}{2}},
        \end{equation}
        位错源开动,在L-C位错锁\index{L-C位错锁}\footnote{L-C位错锁与双滑移有关,第一阶段为单滑移,L-C位错锁数量不会增加。}附近发生塞积,位错的密度有少量增加,切应力增加。
        假设不动位错的数量为$N''$,单位面积不动位错前塞积的位错数目是有上限的,用$n$表示
        塞积的总位错数为$N''*n$,所以切应力为
        \begin{equation}
            \tau=\alpha\mu nb\sqrt{N''},
        \end{equation}
        此时的变形为
        \begin{equation}
            \gamma=\rho Sb=N''L(nb),
        \end{equation}
        其中$L$为位错长度。
        
        此时的加工硬化率为
        \begin{equation}
            \theta=\frac{\dif\sigma}{\dif \varepsilon}=\alpha\mu\frac{1}{L}\cdot N''
        \end{equation}

    \subsection{加工硬化的第二阶段的理论}
        位错开始发生双滑移和多滑移,位错塞积在L-C位错锁前形成新的位错环。由于仍然是塞积在位错锁,
        所以塞积的上限一定$n$不变,而位错锁前形成新的位错锁,所以$N$增加,也可以说是被激活。
        此时产生的形变的增加为
        \begin{equation}
            \dif \gamma=L(nb)\dif N,
        \end{equation}
        位错的长度为
        \begin{equation}
            L=\frac{\Lambda}{\gamma-\gamma_2},
        \end{equation}
        $\gamma_2$为第二阶段切应变分界点,将$L$带回后积分
        \begin{equation}
            \int(\gamma-\gamma_2)\dif\gamma=\int\Lambda(nb)\dif N,
        \end{equation}
        所以形变变化量为
        \begin{equation}
            \gamma-\gamma_2=\sqrt{2knb\Lambda N},
        \end{equation}
        此时的切应力为
        \begin{equation}
            \tau=\alpha\mu(nb)\sqrt{N},
        \end{equation}
        此时的加工硬化率为:
        \begin{equation}
            \theta_2=\frac{\tau}{\gamma-\gamma_2}=\alpha\mu\sqrt{\frac{nb}{2\Lambda}},
        \end{equation}
    \subsection{第三阶段的位错理论}
        螺位错的交滑移,正号和负号的位错相互抵消,使得加工硬化减弱。
    \subsection{包辛格效应}
        金属在正向加载后然后反向加载,会导致屈服应力降低,在单晶体中不容易观察到,由于这一现象与
        晶界的位错塞积和残余应力有关,因此在多晶体中较为明显。

        比如金属处在高周疲劳情况,屈服应力会不断降低,最后断裂。
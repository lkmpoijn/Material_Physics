\chapter{晶体缺陷}
    \section{问题}
    \begin{itemize}
        \item[1] 从几何形态上说,晶体缺陷分哪几类,试举几个典型例子,位错与空位之间有何相互作用?
        \item[2] 为何为点缺陷的形成能和迁移能,为什么说点缺陷是一种热力学平衡缺陷?一定温度下空位平衡浓度的表达式是什么,有什么特点?
        \item[3] 什么叫非平衡点缺陷?试举例说明它的产生方法?
        \item[4] 什么叫位错,位错密度的表达式是什么?
        \item[5] 试总结与比较刃型,螺型及混合位错下列方面的异同(从结构类型,柏式矢量,应力场特点,应变能和张力方面进行描述)。
        \item[6] 何谓柏格斯矢量,柏氏矢量的特点及确定方法。单位长位错的应变能及张力近似表达式是什么?
        \item[7] 为何要引入位错的点阵模型,它能说明什么问题。什么叫位错线宽度,位错中心能量?定性叙述位错的 Pelels模型。
        \item[8] 计算位错运动引起的变形及位错引起的晶体弯曲?
        \item[9] 什么叫作用在位错上的力,这力的大小,方向及特点是什么(分别讨论滑移力,攀移力,化学力),位错受力的 peach公式及由此计算位错在应力场中受到的滑移和攀移力。 
        \item[10] 分类叙述不同类型平行位错线间相互作用力大小,方向及特点?何为位错塞积?
        \item[11] 不同的两条位错相遇时会发生哪些现象?位错反应发生的基本条件是什么?
        \item[12] 不同面位错相互切割后会在各位错线上造成什么后果?什么叫割阶以及它们的运动方式?
        \item[13] 试述位错FR源的动作原理及动作应力表达式?
        \item[14] 什么叫层错?简述面心立方晶体中的不全位错(半,偏位错),扩展位错?什么叫全位错和不动位错?试比较全位错和不全位错的异同点。
        \item[15] 为何说晶界有5个自由度,什么叫扭转晶界,倾转晶界及亚品界?
        \item[16] 晶界为何具有能量,近代关于晶界结构的基本看法是什么?
    \end{itemize}
\chapter{晶体缺陷与强化}
    \section{考试安排}
        考试成绩80\%。
        5个名词、4个简答、2个辨析,说明对错和理由,作图题
        2道大题关于扩散和凝固。

        点缺陷7分、名词解释、线缺陷大约15分、强化8分、断裂10分、单晶滑移8分、回复与再结晶10分。
    \section{问题}
    \begin{itemize}
        \item[1] 从几何形态上说,晶体缺陷分哪几类,试举几个典型例子,位错与空位之间有何相互作用?
        \item[2] 为何为点缺陷的形成能和迁移能,为什么说点缺陷是一种热力学平衡缺陷?一定温度下空位平衡浓度的表达式是什么,有什么特点?
        \item[3] 什么叫非平衡点缺陷?试举例说明它的产生方法?
        \item[4] 什么叫位错,位错密度的表达式是什么?
        \item[5] 试总结与比较刃型,螺型及混合位错下列方面的异同(从结构类型,柏式矢量,应力场特点,应变能和张力方面进行描述)。
        \item[6] 何谓柏格斯矢量,柏氏矢量的特点及确定方法。单位长位错的应变能及张力近似表达式是什么?
        \item[7] 为何要引入位错的点阵模型,它能说明什么问题。什么叫位错线宽度,位错中心能量?定性叙述位错的 Pelels模型。
        \item[8] 计算位错运动引起的变形及位错引起的晶体弯曲?
        \item[9] 什么叫作用在位错上的力,这力的大小,方向及特点是什么(分别讨论滑移力,攀移力,化学力),位错受力的 peach公式及由此计算位错在应力场中受到的滑移和攀移力。 
        \item[10] 分类叙述不同类型平行位错线间相互作用力大小,方向及特点?何为位错塞积?
        \item[11] 不同的两条位错相遇时会发生哪些现象?位错反应发生的基本条件是什么?
        \item[12] 不同面位错相互切割后会在各位错线上造成什么后果?什么叫割阶以及它们的运动方式?
        \item[13] 试述位错FR源的动作原理及动作应力表达式?
        \item[14] 什么叫层错?简述面心立方晶体中的不全位错(半,偏位错),扩展位错?什么叫全位错和不动位错?试比较全位错和不全位错的异同点。
        \item[15] 为何说晶界有5个自由度,什么叫扭转晶界,倾转晶界及亚品界?
        \item[16] 晶界为何具有能量,近代关于晶界结构的基本看法是什么?
    \end{itemize}
    关于固溶体和扩散和相变的问题:
    \begin{enumerate}
        \item[1] 固溶度的影响因素,各因素的机理?
        \begin{enumerate}
            \item 电子浓度,Hume-Rothery定律
            \item 尺寸、
            \item 化学亲合力、
        \end{enumerate} 
        \item[2] 系统形成有序固溶体的条件是什么
        \begin{enumerate}
            \item 动力:内能降低,倾向于异种原子成键;
            \item 阻力:组态熵
        \end{enumerate} 
        \item[3] 中间相的类型
        \begin{enumerate}
            \item 电子化合物
            \item 正常家化合物
            \item 尺寸因素化合物
            \item 脆性相
        \end{enumerate} 
        \item[4] 元素的负电性是什么含义,如何作用形成化合物?
        \item[5] 原子成功跃迁的条件,影响因素是什么?
        \begin{enumerate}
            \item 与熔点有关,熔点越高,原子越稳定,不易跃迁。
        \end{enumerate} 
        \item[6] 扩散过程的两个基本方程,简述内容
        \begin{enumerate}
            \item 菲克第一定律:解释各个物理量含义和单位;
            \item 菲克第二定律
        \end{enumerate} 
        \item[7] 写出扩散过程的决定因素?
        \begin{enumerate}
            \item 化学位梯度
        \end{enumerate} 
        \item[8] 扩散热激活能和影响因素?
        \begin{enumerate}
            \item 扩散系数与温度的关系为$D=D_0\exp\left( -\frac{Q}{RT} \right)$,其中$Q$为扩散激活能;
            \item 扩散激活能与成分、结构、与温度无关。
        \end{enumerate} 
        \item[9] 扩散系数的影响因素?
        \begin{enumerate}
            \item $D=\frac{1}{6}\alpha^2\Gamma=\frac{1}{6}\alpha^2Zp_v\omega$,
            \item 温度
            \item 成分
            \item 结构
        \end{enumerate} 
        \item[10] 卡根达二效应、达肯方程、原始焊接平面、matano面和标记面分意义
        \item[11] 在\ce{Fe-C}相图中,温度高于\SI{923}{\celsius}和在\SIrange{727}{912}{\celsius}之间两种情况下渗碳,碳浓度曲线有何不同?画出来,如何证明两相区($\alpha+\gamma$)不存在?
        \item[12] 给出钍在钨中的体扩散系数、晶粒间扩散系数和表面扩散系数的大小顺序,说明该顺序与温度的关系?
        \item[13] 为什么会出现上坡扩散?是否与菲克定律矛盾?
        \item[14] 影响扩散的因素有那些
        \item[15] 简述间隙固溶体和间隙相?
        \item[16] 形成有序固溶体的阻力是什么?有序-无序转变是不是相变?
        \item[17] 界面能的来源有哪些?
        \item[18] 简述形核的类型和特点,影响形核的因素有什么?
        \item[19] 什么叫脱溶惯序?为什么会发生?
        \item[20] 什么叫正脱溶、负脱溶?举例说明
        \item[21] 从比热-温度曲线说明预沉淀过程。
        \item[22] 什么是GP区?简述特点
        \item[23] 为什么GP区有片状球状,影响因素是什么?
        \item[24] 新相长大的途径有什么?影响因素是什么?
        \item[25] 沉淀相为什么会粗化?其粗化过程是什么?
        \item[26] 发生调幅分解的条件是什么?
        \item[27] 马氏体形貌的控制因素是什么?
        \item[28] 简述马氏体相变,是不是一级相变,为什么?
        \item[29] 马氏体相变类型有哪些?
        \item[30] 共析转变过程,其特点是什么?
    \end{enumerate}
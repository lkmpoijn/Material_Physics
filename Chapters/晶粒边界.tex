\chapter{晶粒边界}
    多晶体材料中,多个晶粒在凝固时的方向不同,因此在边界处的排列方式需要研究,将两个晶粒之间的边界称为晶界\index{晶界}。
    它是把结构相同但相位不同的两个晶粒分隔开的面状晶格缺陷,是本课程中除了层错以外的另一种面缺陷。

    如果吧晶界看作两个晶粒由于取向差的不同造成了晶界,可以发现,界面在空间的方程有2个自由度,而取向差可以认为是有一个晶粒相对与另一个晶粒进行了旋转,这样,可以说
    旋转轴这条直线的方程中有2个自由度,最后旋转角度为第5个自由度:
    \begin{itemize}
        \item[1] 位相差$\theta$;
        \item[2] 发生位相差$\theta$的转动轴的方向余弦,其中仅有两个是独立的量;
        \item[3]  晶界面法线的方向余弦(其中任意二个),这个方向是用来表示晶界在空间取向的。 
    \end{itemize}
    概括起来,就是产生位相差的转动角$\theta$,表示转动轴上单位矢量$\vec{U}$方位的两个参数以及表示晶界发现单位矢量$n$的两个参数。
    假如直到了这些参变量和晶型,就可以确定晶界的位错模型。
    \section{晶界结构}
        一组平行排列的直的刃型位错的稳定平衡位置是沿$y$轴成了一条直线排列,形成位错墙,而形成这一位错墙的原因是墙的两侧有着较大的取向差。
        \subsection{小角度晶界}
            简单晶界有两种类型:倾斜晶界和扭转晶界。设$U$是发生位相差的相对旋转轴上的单位矢量,$n$是晶界面法线上的单位矢量,则纯粹倾转晶界的条件是    
            \begin{equation}
                U\cdot n=0,
            \end{equation}
            如果晶界面与产生取向差的旋转轴垂直,即$U\parallel n$,就构成了简单的扭转晶界。
            \subsubsection{倾转晶界}
                假设两个简单立方晶体具有相同的$[001]$轴,它们之间的位向差是绕着共同轴相对转动$\theta$角而产生的,两个晶粒的截面是一个对称面,
                都和$(100)$面平行。两个晶体以这种方式连接必然导致连接区域的畸变,而且弹性变形区将扩展到足以松弛晶界的应力集中。除了弹性形变还需要一些竖直
                的原子面终止在晶界上,形成刃型位错,其柏式矢量基本都是$[100]$平移矢量,而柏式矢量$\vec{b}$、位错间距$D$和位相差$\theta$的关系为:
                \begin{equation}
                    \frac{b}{D}=\theta,
                \end{equation}
                当$\theta$小于\ang{15}时,为小角晶界,大于\ang{15}时,为大角晶界。
            \subsubsection{扭转晶界}
        \subsection{大角度晶界}
            对于大角晶界仍然没有得到完美的研究结果,此处主要介绍当前的研究进展。

            过冷液体模型认为大叫晶界是几层原子排雷而成,与过冷液体类似,呈非晶态。但是过冷液体在热力学上不稳定,而晶界存在符合平衡条件。
            另外这一模型认为晶界层上将有两个固液界面,这是不能实现的。

            之后又提出了小岛模型等,都不能解释晶界结构。目前较为有效的模型有重合位置电子模型(CSL),假设一下特殊位向的晶界中,有一些原子同属于两边晶粒的格点,
            并自身形成超格点点阵。模型认为大角晶界由约两原子直径厚的对拍和错排区后才,重合位置点阵和大角晶界的关系:
            \begin{itemize}
                \item[1] 重合关系只出现在某些特定的晶界上,晶界总处于重合点阵的最密排面上,而且能量最低厚度很小,长程应变场可以忽略不计;
                \item[2] 晶界与重合电子的最密排面间有一个小角度时,为了使晶界在重合点阵的最密排面上有最大的面积起见,便会产生阶。阶也不具有长程应变场,但如果在晶界上加一适当的应力,它可能成为位错的增殖源;
                \item[3] 与理想重合位置位向稍有偏离的晶界,可以用一个重合位置晶界同一与它在同一平面上的晶界位错网络叠加在一起来描述。一般这种晶界位错的柏氏矢量较晶格位错的为小,故有次位错之称。
            \end{itemize}

            后来又提出了O点阵的概念,O点阵的结点是指在点上看各自晶格近邻关系是相同的,只差一个转角的点,不一定是原子占据的点。
    \section{晶界能量}
        晶界能量来源于两个晶粒边界上很多原子从晶格的正常位置移动出来,并且在附近晶体中引起畸变。我们定义单位面积所对应的能量增加量为晶界能量\index{晶界能量}。
        \subsection{小角度晶界的能量}
            在晶界的各种性质中,晶界能是很重要的物理量,目前关于小角晶界能的计算有
            很多方法。下面介绍一种简明近似的方法,以对称倾斜晶界为例。单位长度位错的能量为:
            \begin{equation}
                W=\frac{\mu b^{2}}{4 \pi(1-v)} \ln \left(\frac{r_{1}}{r_{0}}\right)+W_{A B},
            \end{equation}
            其中,$W_{AB}$为刃型位错中心能,$r_1$为位错弹性应力场所及的距离,大小为亚晶尺寸,$r_0$为位错核心区。

            在单位长度内的位错数量为$1/D$,$D$为位错间距,
            根据
            \begin{equation}
                \frac{1}{D}=\frac{\theta}{b},
            \end{equation}
            另$b=r_0$,$r_1=D$,位错的能量可以写作
            \begin{equation}
                \begin{split}
                    W&=\frac{\theta}{b}\left[\frac{\mu b^{2}}{4 \pi(1-v)} \ln \left(\frac{1}{\theta}\right)+W_{A B}\right]\\
                    &=\frac{\mu b \theta}{4 \pi(1-v)} \ln \left(\frac{1}{\theta}\right)+\frac{\theta}{b} W_{A B}\\
                    &=E_{0} \theta[A-\ln \theta],
                \end{split}
            \end{equation}
            
        \subsection{大角度晶界的能量}
            由于大角晶界结构未知,可以使用测量的方法,假设三个晶界相较于一个公共的交线,平衡的条件为
            \begin{equation}
                \mathrm{E}_{1} / \sin \psi_{1=} \mathrm{E}_{2} / \sin \psi_{2}=\mathrm{E}_{3} / \sin \psi_{3};
            \end{equation}
            一般测量时习惯将两大角晶界能当作不变的参值,而改变第三晶界的旋转角。
            也可用表面沟槽法,表面张力为
    \section{晶界的运动}
        \subsection{小角晶界的移动}
            晶界的应力感生迁移,晶界受到的力为
            \begin{equation}
                P=\tau\cdot b=\theta\cdot\tau,
            \end{equation}
            晶界的迁移率为$B$,驱动力$F$,则迁移速度为
            \begin{equation}
                v=F\cdot B,
            \end{equation}
            驱动力
            \begin{equation}
                F=\frac{\dif \mu}{\dif x},
            \end{equation}
            也就是反化学位梯度的方向\footnote{扩散中原子的运动的方向为化学位的梯度方向}。
            

\chapter{合金强化}
    决定材料强度的关键因素是原子之间的结合力以及位错。原子的结合力与材料的原子有关,这非常难控制,
    但是位错是人类可以控制的重要因素,因此位错是近代金属领域的最大成果。

    \section{强化金属的途径}
        根据位错运动出发,强化的方式主要有两种。
        \subsection{金属的极限强度}
            金属的极限强度约为\SI{6000}{\MPa},然而实际的材料仅仅为极限的$\frac{1}{10}$,
            想要达到这样的强度就必须实现理想晶体,使晶体满足刚性滑移假设。在实际上通常是把位错
            滑移出晶体,但是一旦晶体中出现位错,性能就急剧下降。
        \subsection{人为增加位错的阻力}
            提升晶格阻力和源动作应力是提升强度的重要方法:通过对金属中各种微结构与位错的相互作用提升强度的方法主要有四种
            \begin{enumerate}
                \item[1] 点缺陷:固溶强化;
                \item[2] 线缺陷:加工硬化,根据切应力公式$\gamma=\rho sb$;
                \item[3] 面缺陷:晶粒细化强化,根据霍尔佩奇公式$\sigma_s=\sigma_0+k\cdot d^{-1/2}$;
                \item[4] 体缺陷:分散强化。
            \end{enumerate}
            在工业中,都是使用综合强化。比如同一个奥氏体中的马氏体片条组,可以视为将奥氏体细化实现强化。
    \section{人为强化机制}
        \subsection{固溶强化}
            溶质原子与气团的相互作用有两种,直接作用和间接作用。

            基本规律是
            \begin{itemize}
                \item 固溶体强度大于纯金属,细固溶体的强度与间隙原子的浓度成线性关系,间隙原子半径越小,强化效果越大;
                \item 溶解度小的元素强化作用较大;
                \item 稀固溶体的强化作用可以叠加;
                \item 间隙原子作用大于置换原子;
                \item 相同的电子浓度,强化效果接近;
                \item 模量相应,加入物质的模量越大,强化效果越大。
            \end{itemize}
        \subsection{固溶强化机制}
            \begin{enumerate}
                \item 弹性相互作用:气团,
                \begin{enumerate}
                    \item 是明显屈服点产生的原因,对位错的启动有作用;
                    \item 溶质原子是间隙原子可以使间隙现象更为明显,在体心立方中比面心立方中更明显;
                    \item 对温度敏感,温度升高,扩散加快;
                    \item 只要有少量溶质就可以起作用。
                \end{enumerate}
                \item 溶质原子分布的摩擦阻力;
                \item 电学交互作用:较弱,位错两侧的溶质原子的库仑力产生拉应力或压应力;
                \item 化学交互作用:较弱;
                \item 有序化:主要在有序化的材料中才会出现,分为长程序和短程序,在有序晶体内产生反向畴界。
            \end{enumerate}
            其中第一点、第四点第五点对位错启动有作用,2、3、5总是对位错运动有作用。
            温度升高,第一类作用下降。
        \subsection{屈服现象与应变时效}
            有两种屈服现象,分为有明显屈服平台和无平台(均匀屈服)两种。
            
            第一种的产生原因是科垂耳气团,在超过上屈服点后应力下降,发生动态再结晶,
            这一阶段称为屈服齿,也叫做吕德斯应变,是形变传播过程中的必然现象。

            均匀屈服没有屈服平台,而是呈凹陷区域。在$\alpha$\ce{Fe}经过除\ce{H}处理后,
            可以发现。一般认为是可动位错的密度的提高,导致屈服凹陷的出现
            \begin{align}
                \gamma=\rho sb,\\
                \varepsilon=\phi\rho sb\\
                \dot{\varepsilon}=\phi\rho\dot{s}b.
            \end{align}
            拉伸时,速度一定,所以只能是位错速度下降,出现凹陷。
        \subsection{分散强化}
            如果材料中出现第二相,形成两相合金。在这两种相之间的界面上的原子排列不再具有晶格完整性。在金属等塑性材料中,这些相界面会阻碍位错的滑移,从而使材料得到强化。这就是分散强化的由来。
            分成弥散强化、时效(沉淀)强化两类。

            当位错经过第二相时,会出现绕过机制,形成位错环。或是根据切过机制,位错切过第二相。
        
    \section{补充}
        \begin{enumerate}
            \item 兰脆现象:当钢铁加热到\SIrange{200}{300}{\degree},钢铁的脆性增加,表面变成兰色,原因是钢铁的动态应变时效。
            \item 预应变:提前拉伸钢铁,使其脱开溶质原子的束缚。
        \end{enumerate}